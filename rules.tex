\documentclass[10pt]{article}

\usepackage{mathtools}
\usepackage{amsmath, amssymb, amsthm}
\usepackage{mathpartir}
\usepackage{geometry}
\usepackage{syntax}
\usepackage{txfonts}
\usepackage{subdepth}

\newcommand{\xto}{\xrightarrow}
\newcommand{\wto}{\rightsquigarrow}
\newcommand{\aless}{\preccurlyeq}
\newcommand{\lto}{\longrightarrow}
\newcommand{\context}{\Gamma;\Sigma\mid\Phi}
\let\emptyset\varnothing

\title{Singleton Documentation}

\theoremstyle{definition}\newtheorem*{theorem}{Theorem}
\theoremstyle{definition}\newtheorem*{definition}{Definition}
\theoremstyle{definition}\newtheorem*{lemma}{Lemma}

\begin{document}
	\maketitle
    \paragraph{Syntax}
    \setlength{\grammarindent}{8em} % increase separation between LHS/RHS 
    \begin{grammar}
        <term> ::= <variable> 
        \vspace{0.8em}
        \alt <number>
        \vspace{0.8em}
        \alt unit
        \vspace{0.8em}
        \alt $\lambda$ <variable> : <type> . <term>
        \vspace{0.8em}
        \alt <term> <term>
        \vspace{0.8em}
        \alt ref <term> 
        \vspace{0.8em}
        \alt ! <term>
        \vspace{0.8em}
        \alt <term> `:=' <term>
        \vspace{0.8em}
        \alt <location>
        \vspace{0.8em}
        \alt mvar <term> 
        \vspace{0.8em}
        \alt \$ <term>
        \vspace{0.8em}
        \alt <term> `=<<' <term>
        \vspace{0.8em}
        \alt \{\;<term>, <term>\;\}
        \vspace{0.8em}
        \alt <term>.1
        \vspace{0.8em}
        \alt <term>.2
        \vspace{0.8em}
        \alt fork <term> <term>

        <value> ::= <number>
        \vspace{0.8em}
        \alt unit
        \vspace{0.8em}
        \alt $\lambda$ <variable> : <type> . <term>
        \vspace{0.8em}
        \alt <location>
        \vspace{0.8em}
        \alt ref <value>
        \vspace{0.8em}
        \alt mvar <value>
        \vspace{0.8em}
        \alt \{\;<value>, <value>\;\}

        <type> ::= Nat
        \vspace{0.8em}
        \alt Unit
        \vspace{0.8em}
        \alt Ref <type>
        \vspace{0.8em}
        \alt Mvar <type>
        \vspace{0.8em}
        \alt <type> $\xto{\varphi}$ <type>
        \vspace{0.8em}
        \alt <type> $\times$ <type>

        <variable> ::= [a-z]

        <number> ::= [1-9][0-9]*
    \end{grammar}

    \paragraph{Kind Inference}
    \begin{mathpar}
        \inferrule{\vdash\mathrm{Unit}\colon\mathbf{U}}{} \and
        \inferrule{\vdash\mathrm{Ref}\ \mathrm{T}\colon\mathbf{A}}{} \and
        \inferrule{\vdash\mathrm{Mvar}\ \mathrm{T}\colon\mathbf{U}}{} \and
        \inferrule{\vdash\mathrm{Nat}\colon\mathbf{U}}{} \and
        \inferrule{\vdash\mathrm{T_1}\xto{\varphi}\mathrm{T_2}\colon \varphi}{} \and
        \inferrule{\vdash\mathrm{T_1}\colon\varphi_1\\ \vdash\mathrm{T_2}\colon\varphi_2}{\vdash \mathrm{T_1}\times\mathrm{T_2}\colon\varphi_1\sqcap\varphi_2}
    \end{mathpar}

    \paragraph{Kind Bound}
    \begin{mathpar}
        \inferrule{\vdash\emptyset\aless\mathbf{U}}{} \and
        \inferrule{\vdash\Sigma\aless\varphi\\ \vdash\mathrm{T}\colon\mathbf{U}}{\vdash\Sigma\cup\{\mathrm{x}\colon\mathrm{T}\}\aless\varphi} \and
        \inferrule{\vdash\Sigma\aless\varphi\\ \vdash\mathrm{T}\colon\mathbf{A}}{\vdash\Sigma\cup\{\mathrm{x}\colon\mathrm{T}\}\aless\mathbf{A}}
    \end{mathpar}

    \paragraph{Context Extension}
    \begin{mathpar}
        \inferrule{\vdash\Gamma;\Sigma,\,*\wto\Gamma;\Sigma}{}\and
        \inferrule{\vdash \mathrm{T}\colon \mathbf{U}\\ \vdash\Gamma_0\cup\{\mathrm{x}\colon \mathrm{T}\};\Sigma_0,\,\Sigma'\wto\Gamma;\Sigma}{\vdash\Gamma_0;\Sigma_0,\,\{\mathrm{x}\colon \mathrm{T}\}\cup\Sigma'\wto\Gamma;\Sigma} \and
        \inferrule{\vdash \mathrm{T}\colon \varphi\\ \vdash\Gamma_0;\Sigma_0\cup\{\mathrm{x}\colon \mathrm{T}\},\,\Sigma'\wto\Gamma;\Sigma}{\vdash\Gamma_0;\Sigma_0,\,\{\mathrm{x}\colon \mathrm{T}\}\cup\Sigma'\wto\Gamma;\Sigma}
    \end{mathpar}

    \paragraph{Type Inference Rules} \mbox{} \\
    \begin{mathpar}
        \inferrule{\context \vdash \mathrm{unit}\colon\mathrm{Unit}}{} \and
        \inferrule{\context \vdash \mathrm{n}\colon \mathrm{Nat}}{} \and
        \inferrule{\mathrm{x}\colon \mathrm{T} \in \Gamma}{\context\vdash \mathrm{x}\colon \mathrm{T}} \and
        \inferrule{\mathrm{x}\colon \mathrm{T} \in \Sigma}{\context \vdash \mathrm{x}\colon \mathrm{T}} \and
        \inferrule{\Gamma;\Sigma_1\mid\Phi \vdash \mathrm{t_1}\colon \mathrm{T_1}\xto{\varphi}\mathrm{T_2}\\ \Gamma;\Sigma_2\mid\Phi \vdash \mathrm{t_2}\colon \mathrm{T_1}}{\Gamma;\Sigma_1\cup\Sigma_2\mid\Phi \vdash \mathrm{t_1}\ \mathrm{t_2}\colon \mathrm{T_2}} \and
        \inferrule{\vdash \Gamma;\Sigma,\,\mathrm{x}\colon \mathrm{T_1} \wto \Gamma';\Sigma' \\ \Gamma';\Sigma'\mid\Phi \vdash \mathrm{t}\colon \mathrm{T_2} \\ \Gamma\vdash\Sigma\aless\varphi}{\context\vdash\lambda \mathrm{x}\colon \mathrm{T_1}.\ \mathrm{t_2}\colon \mathrm{T_1}\xto{\varphi}\mathrm{T_2}} \and
        \inferrule{\context \vdash \mathrm{t}\colon \mathrm{T}}{\context \vdash \mathrm{ref}\ \mathrm{t}\colon  \mathrm{Ref}\ \mathrm{T}} \and
        \inferrule{\context \vdash\mathrm{t}\colon \mathrm{Ref}\ \mathrm{T}}{\context \vdash!\,\mathrm{t}\colon \mathrm{T}}\and
        \inferrule{\context \vdash\mathrm{t_1}\colon \mathrm{Ref}\ \mathrm{T}\\ \context \mathrm{t_2}\colon \mathrm{T}}{\context\vdash \mathrm{t_1}:=\mathrm{t_2}\colon \mathrm{Unit}} \and
        \inferrule{\context \vdash\mathrm{t}\colon \mathrm{T}}{\context \vdash\mathrm{mvar}\ \mathrm{t}\colon\mathrm{Mvar}\ \mathrm{T}} \and
        \inferrule{\context \vdash\mathrm{t}\colon \mathrm{Mvar}\ \mathrm{T}}{\context \vdash\$\,\mathrm{t}\colon \mathrm{T}} \and
        \inferrule{\context \vdash\mathrm{t_1}\colon \mathrm{Mvar}\ \mathrm{T}\\ \context \vdash\mathrm{t_2}\colon \mathrm{T}}{\context\vdash\mathrm{t_1}=<<\mathrm{t_2}\colon\mathrm{Unit}} \and
        \inferrule{\Phi(\mathrm{l})=\mathrm{T}}{\context \vdash \mathrm{l}\colon \mathrm{T}}\and
        \inferrule{\context \vdash\mathrm{t_1}\colon \mathrm{T_1}\\\context \vdash\mathrm{t_2}\colon \mathrm{T_2}}{\context\vdash\{\,\mathrm{t_1}, \mathrm{t_2}\,\}\colon\mathrm{T_1}\times\mathrm{T_2}} \and
        \inferrule{\context \vdash\mathrm{t}\colon\mathrm{T_1}\times\mathrm{T_2}}{\context \vdash\mathrm{t}.\mathrm{1}\colon\mathrm{T_1}} \and
        \inferrule{\context \vdash\mathrm{t}\colon\mathrm{T_1}\times\mathrm{T_2}}{\context \vdash\mathrm{t}.\mathrm{2}\colon\mathrm{T_2}} \and
        \inferrule{\Gamma;\Sigma_1\mid\Phi \vdash\mathrm{t_1}\colon\mathrm{T_1}\\ \Gamma;\Sigma_2\mid\Phi \vdash\mathrm{t_2}\colon\mathrm{T_2}}{\Gamma;\Sigma_1\cup\Sigma_2\mid\Phi \vdash\mathrm{fork}\ \mathrm{t_1}\ \mathrm{t_2}\colon\mathrm{T_1}\times\mathrm{T_2}} \and
        \inferrule{\context \vdash\mathrm{t}\colon\mathrm{T}\xto{\varphi}\mathrm{T}}{\context \vdash \mathrm{fix}\ \mathrm{t}\colon\mathrm{T}}
    \end{mathpar}

    \paragraph{Evaluation Rules}
    \begin{mathpar}
        \inferrule{\mathrm{t_1}\mid\mu\lto \mathrm{t_1}'\mid\mu'}{\mathrm{t_1}\ \mathrm{t_2}\mid\mu\lto \mathrm{t_1}'\ \mathrm{t_2}\mid\mu'}\and
        \inferrule{\mathrm{t_2}\mid\mu\lto \mathrm{t_2}'\mid\mu'}{\mathrm{v_1}\ \mathrm{t_2}\mid\mu\lto \mathrm{v_1}\ \mathrm{t_2}'\mid\mu'}\and
        \inferrule{(\lambda \mathrm{x}\colon\mathrm{T_1}.\ \mathrm{t})\ \mathrm{v}\mid\mu\lto[\mathrm{x}\mapsto \mathrm{v}]\,\mathrm{t}\mid\mu}{} \and

        \inferrule{\mathrm{l}\not\in dom(\,\mu)}{\mathrm{ref}\ \mathrm{v_1}\mid\mu\lto\mathrm{l}\mid(\,\mu, \mathrm{l}\mapsto\mathrm{v_1})} \and
        \inferrule{\mathrm{t_1}\mid\mu\lto\mathrm{t_1'}\mid\mu'}{\mathrm{ref}\ \mathrm{t_1}\mid\mu\lto\mathrm{ref}\ \mathrm{t_1'}\mid\mu'} \and
        \inferrule{\mu\,(\mathrm{l})=\mathrm{v}}{!\,\mathrm{l}\mid\mu\lto\mathrm{v}\mid\mu} \and
        \inferrule{\mathrm{t_1}\mid\mu\lto\mathrm{t_1'}\mid\mu'}{!\,\mathrm{t_1}\mid\mu\lto !\,\mathrm{t_1'}\mid\mu'}\and
        \inferrule{\mathrm{l}:=\mathrm{v}\mid\mu\lto\mathrm{unit}\mid[\mathrm{l}\mapsto\mathrm{v}]\ \mu}{} \and
        \inferrule{\mathrm{t_1}\mid\mu\lto\mathrm{t_1'}\mid\mu'}{\mathrm{t_1}:=\mathrm{t_2}\mid\mu\lto\mathrm{t_1'}:=\mathrm{t_2}\mid\mu'} \and
        \inferrule{\mathrm{t_2}\mid\mu\lto\mathrm{t_2'}\mid\mu'}{\mathrm{v_1}:=\mathrm{t_2}\mid\mu\lto\mathrm{v_1}:=\mathrm{t_2'}\mid\mu'} \and

        \inferrule{\mathrm{l}\not\in dom(\,\mu)}{\mathrm{mvar}\ \mathrm{v_1}\mid\mu\lto\mathrm{l}\mid(\,\mu, \mathrm{l}\mapsto\mathrm{v_1})} \and
        \inferrule{\mathrm{t_1}\mid\mu\lto\mathrm{t_1'}\mid\mu'}{\mathrm{mvar}\ \mathrm{t_1}\mid\mu\lto\mathrm{mvar}\ \mathrm{t_1'}\mid\mu'} \and
        \inferrule{\mu\,(\mathrm{l})=\mathrm{v}}{\$\,\mathrm{l}\mid\mu\lto\mathrm{v}\mid\mu} \and
        \inferrule{\mathrm{t_1}\mid\mu\lto\mathrm{t_1'}\mid\mu'}{\$\,\mathrm{t_1}\mid\mu\lto\$\,\mathrm{t_1'}\mid\mu}\and
        \inferrule{\mathrm{l}=<<\mathrm{v}\mid\mu\lto\mathrm{unit}\mid[\mathrm{l}\mapsto\mathrm{v}]\ \mu}{} \and
        \inferrule{\mathrm{t_1}\mid\mu\lto\mathrm{t_1'}\mid\mu'}{\mathrm{t_1}=<<\mathrm{t_2}\mid\mu\lto\mathrm{t_1'}=<<\mathrm{t_2}\mid\mu'} \and
        \inferrule{\mathrm{t_2}\mid\mu\lto\mathrm{t_2'}\mid\mu'}{\mathrm{v_1}=<<\mathrm{t_2}\mid\mu\lto\mathrm{v_1}=<<\mathrm{t_2'}\mid\mu'} \and

        \inferrule{\{\,\mathrm{v_1}, \mathrm{v_2}\,\}.1\lto\mathrm{v_1}}{} \and
        \inferrule{\{\,\mathrm{v_1}, \mathrm{v_2}\,\}.2\lto\mathrm{v_2}}{} \and
        \inferrule{\mathrm{t}\lto\mathrm{t'}}{\mathrm{t.1}\lto\mathrm{t'.1}}{} \and
        \inferrule{\mathrm{t}\lto\mathrm{t'}}{\mathrm{t.2}\lto\mathrm{t'.2}}{} \and
        \inferrule{\mathrm{t_1}\lto\mathrm{t'_1}}{\{\,\mathrm{t_1}, \mathrm{t_2}\,\}\lto\{\,\mathrm{t'_1}, \mathrm{t_2}\,\}}{} \and
        \inferrule{\mathrm{t_2}\lto\mathrm{t'_2}}{\{\,\mathrm{v_1}, \mathrm{t_2}\,\}\lto\{\,\mathrm{v_1}, \mathrm{t'_2}\,\}}{} \and

        \inferrule{\mathrm{fork}\ \mathrm{v_1}\ \mathrm{v_2}\lto\{\,\mathrm{v_1}, \mathrm{v_2}\,\}}{}\and
        \inferrule{\mathrm{t_1}\lto\mathrm{t'_1}}{\mathrm{fork}\ \mathrm{t_1}\ \mathrm{t_2}\lto \mathrm{fork}\ \mathrm{t'_1}\ \mathrm{t_2}} \and
        \inferrule{\mathrm{t_2}\lto\mathrm{t'_2}}{\mathrm{fork}\ \mathrm{t_1}\ \mathrm{t_2}\lto \mathrm{fork}\ \mathrm{t_1}\ \mathrm{t'_2}} \and

        \inferrule{\mathrm{fix}\ (\lambda \mathrm{x}\colon\mathrm{T_1}.\mathrm{t_2})\lto [\mathrm{x}\mapsto(\mathrm{fix}\ (\lambda \mathrm{x}\colon\mathrm{T_1}.\mathrm{t_2}))]\ \mathrm{t_2}}{} \and
        \inferrule{\mathrm{t_1}\lto\mathrm{t'_1}}{\mathrm{fix}\ \mathrm{t_1}\lto\mathrm{fix}\ \mathrm{t'_1}}
    \end{mathpar}
    
    \newpage
    \begin{definition}
        A store $\mu$ is said to well typed with respect to a typing context $\Gamma;\Sigma$ and a store typing $\Phi$,
        written $\context\vdash\mu$, iff $dom(\mu) = dom(\Phi)$ and $\context\vdash \mu(\mathrm{l})\colon\Phi(\mathrm{l}),\ \forall \mathrm{l} \in dom(\mu)$.
    \end{definition}
    \begin{theorem}[{\it Preservation}]
        if 
        \begin{itemize}
            \item[] $\context\vdash\mathrm{t}\colon\mathrm{T}$
            \item[] $\context\vdash\mu$
            \item[] $\mathrm{t}\mid\mu\lto\mathrm{t'}\mid\mu'$
        \end{itemize}
        then for some $\Phi'\supseteq\Phi$,
        \begin{itemize}
            \item[] $\Gamma;\Sigma\mid\Phi'\vdash\mathrm{t}\colon\mathrm{T}$
            \item[] $\Gamma;\Sigma\mid\Phi'\vdash\mu'$
        \end{itemize}
    \end{theorem}
    \begin{proof}
        For $\context\vdash\mathrm{t_1}\colon\mathrm{T}$ and $\mathrm{t_1}\mid\mu\lto\mathrm{t_2}\mid\mu$, it can be given by the induction on the structure of type derived tree T. Here are the induction cases:
        \begin{itemize}
            \item $\mathrm{t_1}$ is in form that $\mathrm{t_{11}}\ \mathrm{t_{12}}$. By the inversion lemma $\mathrm{t_{11}}$ must have type $\mathrm{T_1}\to\mathrm{T}$ while $\mathrm{t_{12}}$ has type $\mathrm{T_1}$.
                Here comes the subcase:
                \begin{itemize}
                    \item In the case that evaluation rule $\mathrm{t_{11}}\ \mathrm{t_{12}} \lto \mathrm{t'_{11}}\ \mathrm{t_{12}}$ is applied, $\mathrm{t'_{11}}$ must have type $\mathrm{T_1}\to\mathrm{T}$ by the induction hypothesis. So $\mathrm{t'_{11}}\ \mathrm{t_{12}}$ has type $\mathrm{T}$, the hypothesis holds.
                    \item In the case that evaluation rule $\mathrm{t_{11}}\ \mathrm{t_{12}} \lto \mathrm{t_{11}}\ \mathrm{t'_{12}}$ is applied, it is similar.
                    \item In the case that E-SUB evaluation rule is applied, $\mathrm{t_{11}}$ is in form that $\lambda \mathrm{x}\colon \mathrm{T_1}.\ \mathrm{t_{13}}$ by the canonical lemma while $\mathrm{t_{13}}$ has type T.
                        And $\mathrm{t_2}$ is in form that $[\mathrm{x}\mapsto\mathrm{t_{12}}]\, \mathrm{t_{13}}$. Note that x and $\mathrm{t_{12}}$ both have type $\mathrm{T_1}$, $[\mathrm{x}\mapsto\mathrm{t_{12}}]\, \mathrm{t_{13}}$ has type T by the substitution lemma. The hypothesis holds.
                \end{itemize}
                \item $\mathrm{t_1}$ is in form that $!\mathrm{t_{11}}$. By the inversion lemma $\mathrm{t_{11}}$ must have type $\mathrm{Ref}\,\mathrm{T}$.
                    \begin{itemize}
                        \item In the case that evaluation rule $!\mathrm{t_{11}}\lto!\mathrm{t'_{11}}$ is applied, $\mathrm{t'_{11}}$ must have type $\mathrm{Ref}\,\mathrm{T}$ by the hypothesis,
                            so $!\mathrm{t'_{11}}$ has type $T$. The hypothesis holds.
                        \item In the case that evaluation rule $!\mathrm{t_{11}}\lto \mathrm{x}$ is applied where $\mathrm{t_{11}} = \mathrm{ref}\ \mathrm{x}$. By the inversion lemma, x must have type $T$. The hypothesis holds.
                    \end{itemize}
                \item $\mathrm{t_1}$ is in form that $@\mathrm{t_{11}}$. It is similar to the previous case.
                \item $\mathrm{t_1}$ is in form that $\mathrm{fork}\ \mathrm{t_{11}}\ \mathrm{t_{12}}$ and $\mathrm{T}$ is $\mathrm{T_1}\times\mathrm{T_2}$ while the type of $\mathrm{t_{11}}$ and $\mathrm{t_{12}}$ are $\mathrm{T_1}$ and $\mathrm{T_2}$ respectively.
                    \begin{itemize}
                        \item In the case that evaluation rule $\mathrm{fork}\ \mathrm{t_{11}}\ \mathrm{t_{12}} \lto \mathrm{fork}\ \mathrm{t'_{11}}\ \mathrm{t_{12}}$. By the induction hypothesis, $\mathrm{t'_{11}}$ has type $\mathrm{T_1}$. So $\mathrm{fork}\ \mathrm{t'_{11}}\ \mathrm{t_{12}}$ has type $\mathrm{T_1}\times\mathrm{T_2}$.
                            The hypothesis holds.
                        \item In the case that evaluation rule $\mathrm{fork}\ \mathrm{t_{11}}\ \mathrm{t_{12}} \lto \mathrm{fork}\ \mathrm{t_{11}}\ \mathrm{t'_{12}}$. It is similar to previous case.
                        \item In the case that evaluation rule $\mathrm{fork}\ \mathrm{t_{11}}\ \mathrm{t_{12}} \lto \{\mathrm{t_{11}},\,\mathrm{t_{12}}$. In this case, both $\mathrm{t_{11}}$ and $\mathrm{t_{12}}$ are values, so the type of result is $\mathrm{T_1}\times\mathrm{T_2}$. The hypothesis holds.
                    \end{itemize}
                \item Ohter cases are trival and similar to the proof in textbook, we omit them here.
        \end{itemize}
    \end{proof}
    \begin{theorem}[{\it Progress}]
        Suppose $\mathrm{t}$ is a closed, well typed term 
        (i.e. $\emptyset;\emptyset\mid\Phi\vdash\mathrm{t}\colon\mathrm{T}$ for some $\Phi$ and $\mathrm{T}$). 
        Then either $\mathrm{t}$ is a value or else, for any store $\mu$ such that $\emptyset;\emptyset\mid\Phi\vdash\mu$, 
        there is some term $\mathrm{t'}$ and store $\mu$ with $\mathrm{t}\mid\mu\lto\mathrm{t'}\mid\mu'$
    \end{theorem}
    \begin{proof}
        The proof can be given by induction on the structure of type derivation tree.
        \begin{itemize}
            \item t is in form that $\mathrm{t_{11}}\ \mathrm{t_{12}}$. By the inversion lemma, $\mathrm{t_{11}}$ has type $\mathrm{T_1}\to\mathrm{T}$ while $\mathrm{t_{12}}$ has type $\mathrm{T_1}$.
                \begin{itemize}
                    \item In the case that $\mathrm{t_{11}}$ is not a value, it can take an evaluation step by induction hypothesis. So the whole term t can take an evaluation step by $E-APP1$.
                    \item It is similar to the case that $\mathrm{t_{12}}$ is not a value. 
                    \item In the case that both $\mathrm{t_{11}}$ and $\mathrm{t_{12}}$ are value, $\mathrm{t_{11}}$ must be $\lambda\ \mathrm{x}\colon\mathrm{T_1}.\,\mathrm{t_{13}}$ by canonical lemma.
                            So E-SUB rule can be applied and the induction hypothesis holds.
                \end{itemize}
            \item t is in form that $!\mathrm{t_{11}}$. 
                \begin{itemize}
                    \item In the case that $\mathrm{t_{11}}$ is not a value, $\mathrm{t_{11}}$ can take an evaluation step. So the whole term can take an evaluation step and the hypothesis holds.
                    \item In the case that $\mathrm{t_{11}}$ is a value, it must be a location $\mathrm{l}$. For every location, it must bind a value in the storage. So the whole term can take an evaluation step to the binded value and the hypothesis holds.
                \end{itemize}
            \item t is in form that fork $\mathrm{t_1}\ \mathrm{t_2}$. If $\mathrm{t_1}$ is not a value, it can take an evaluation step and thus the whole term can take a step.
                It is similar to that $\mathrm{t_2}$ is not a value. If both $\mathrm{t_1}$ and $\mathrm{t_2}$ are value, the whole term can take a step to a pair. The hypothesis holds.
            \item Other cases are trival and similar to the proof in textbook, we omit them here.
        \end{itemize}
    \end{proof}
    \begin{lemma}[{\it Unique reference lemma}]
        Suppose $\mathrm{t}$ is a closed, well typed term. All reference to involved locations are unique during evaluation.
    \end{lemma}
    \begin{proof}
        The proof can be given by induction on the structure of type derivation tree.
        \begin{itemize}
            \item t is in form that $\mathrm{t_1}\ \mathrm{t_2}$. By the inversion lemma, $\mathrm{t_1}$ has type $\mathrm{T_1}\to\mathrm{T}$ and $\mathrm{T_1}$ respectively.
                According to the induction hypothesis, all locations involved in $\mathrm{t_1}$ are unique in $\mathrm{t_1}$ and it also holds for $\mathrm{t_2}$. If there
                is a reference $\mathrm{r}$ with kind $\mathbf{A}$ occurs in both $\mathrm{t_1}$ and $\mathrm{t_2}$. Note that all reference with kind $\mathbf{A}$ all reside in type context $\Sigma$ and the type context $\Sigma$ is splited into two parts $\Sigma_1$ and $\Sigma_2$ with no overlaps.
                If $\mathrm{r}$ is divided into $\Sigma_1$, then $\mathrm{t_2}$ is ill typed due to the lack of context for $\mathrm{r}$. So dose the case that $\mathrm{r}$ is divied into $\Sigma_2$.
            \item t is in form that $\mathrm{fork}\ \mathrm{t_1}\ \mathrm{t_2}$. It is similar to previous case.
            \item Other cases are trival, we omit it here.
        \end{itemize}
    \end{proof}
    \begin{definition}[{\it Forked state}]
        A state $\mathrm{t}\mid\mu$ is said to forked state if $\mathrm{t}\mid\mu\lto\mathrm{t_1}\mid\mu_1$ and $\mathrm{t}\mid\mu\lto\mathrm{t_2}\mid\mu_2$ hold while $\mathrm{t_1}$ does not equal to $\mathrm{t_2}$.
    \end{definition}
    \begin{theorem}[{\it Race condition free}]
        Suppose $\mathrm{t}$ is a closed, well typed term. There dose not exist a fork state in which the next steps access same storage location whose kind is $\mathbf{A}$ during the evaluation process of t.
    \end{theorem}
    \begin{proof}
        The proof can be given by induction on the structure of type derivation tree.
        \begin{itemize}
            \item t is in form that $\mathrm{t_1}\ \mathrm{t_2}$. By the inversion lemma, $\mathrm{t_1}$ has type $\mathrm{T_1}\to\mathrm{T}$ and $\mathrm{t_2}$ has type $\mathrm{T_1}$.
                \begin{itemize}
                    \item $\mathrm{t_1}$ is value and $\mathrm{t_2}$ is not value. Thus, $\mathrm{t_2}\mid\mu$ is forked state and the hypothesis holds during the evaluation of $\mathrm{t_2}$ by induction hypothesis.
                    \item $\mathrm{t_2}$ is not value and $\mathrm{t_2}$ is not value. Thus, $\mathrm{t_1}\mid\mu$ is forked state and the hypothesis holds during the evaluation of $\mathrm{t_1}$ by induction hypothesis.
                    \item Other case can not occur.
                \end{itemize}
            \item t is in form that $\mathrm{fork}\ \mathrm{t_1}\ \mathrm{t_2}$, in which $\mathrm{t_1}$ has type $\mathrm{T_1}$, $\mathrm{t_2}$ has type $\mathrm{T_2}$ and $\mathrm{t}$ has type $\mathrm{T_1}\times\mathrm{T_2}$
                \begin{itemize}
                    \item $\mathrm{t_1}$ is value and $\mathrm{t_2}$ is not value. In this case, $\mathrm{t_2}\mid\mu$ is forked state and the hypothesis holds for $\mathrm{t_2}$ by induction hypothesis.
                    \item $\mathrm{t_2}$ is vlaue and $\mathrm{t_1}$ is not value. It is similar to previous case.
                    \item Neither $\mathrm{t_1}$ or $\mathrm{t_2}$ is value.
                        \begin{itemize}
                            \item $\mathrm{t_1}\mid\mu$ is forked state. Then, $\mathrm{t_1}\mid\mu\lto\mathrm{t_{11}}\mid\mu_{11}$ and $\mathrm{t_1}\mid\mu\lto\mathrm{t_{12}}\mid\mu_{12}$. By the induction hypothesis, $\mathrm{t_{11}}$ and $\mathrm{t_{12}}$ does not access same storage.
                                Thus, $\mathrm{fork}\ \mathrm{t_1}\ \mathrm{t_2}\mid\mu\lto\mathrm{fork}\ \mathrm{t_{11}}\ \mathrm{t_2}\mid\mu_{11}$ and $\mathrm{fork}\ \mathrm{t_1}\ \mathrm{t_2}\mid\mu\lto\mathrm{fork}\ \mathrm{t_{12}}\ \mathrm{t_2}\mid\mu$ hold the hypothesis.
                            \item $\mathrm{t_2}\mid\mu$ is forked state. It is similar to previous case.
                            \item Neither $\mathrm{t_1}\mid\mu$ or $\mathrm{t_2}\mid\mu$ is forked state. According the unique reference lemma, every storage location with kind $\mathrm{A}$ has an unique reference during the evaluation.
                                Thus, either $\mathrm{t_1}$ or $\mathrm{t_2}$ is a reference to a specific location. It can not happen that $\mathrm{t_1}$ and $\mathrm{t_2}$ are the reference to a same location. So $\mathrm{fork}\ \mathrm{t_1}\ \mathrm{t_2}\lto\mathrm{fork}\ \mathrm{t'_1}\ \mathrm{t_2}$
                                and $\mathrm{fork}\ \mathrm{t_1}\ \mathrm{t_2}\lto\mathrm{fork}\ \mathrm{t_1}\ \mathrm{t'_2}$ hold the hypothesis.
                        \end{itemize}
                \end{itemize}
            \item Other cases are trival, we omit the proof.
        \end{itemize}
    \end{proof}
\end{document}

